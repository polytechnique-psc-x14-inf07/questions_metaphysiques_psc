\documentclass[a4paper, 11pt]{article}
\usepackage[utf8]{inputenc}
\usepackage[T1]{fontenc}
\usepackage[french]{babel}

\usepackage{lmodern}
\usepackage{textcomp}
\usepackage{ifthen} \usepackage{amsmath} \usepackage{amsfonts} \usepackage{amssymb} \usepackage{graphicx}
\usepackage{enumitem}
\usepackage{multicol}
\usepackage[pagenumber]{polytechnique}
\usepackage[pdftex=true,
			colorlinks=true,
			linkcolor=black,
			filecolor=red,
			urlcolor=blue,
			bookmarks=true,
			bookmarksopen=true]{hyperref}


\title{Questions métaphysiques}
\author{Pierrick \textsc{Allègre} \\
		Gustavo \textsc{Castro} \\
		Clément \textsc{Durand} \\
		Felipe \textsc{Garcia} \\
		Alexandre \textsc{Harry} \\
		Francisco \textsc{Ribeiro Eckhardt Serpa} \\
		Pierre-Alexandre \textsc{Thomas} \\
		Guillaume \textsc{Vizier}}
\subtitle{Projet Scientifique Collectif}
\date{\today}

\begin{document}
%\pagestyle{} %ou plain, headings, empty
\maketitle
%\tableofcontents

%\noindent\textcolor{bleu303}{\rule{\textwidth}{\epaisseurtrait}}

Pour ajouter une question, mettre dans le code :
\begin{verbatim}
\ask[<auteur>]{<question>}
\end{verbatim}

Pour répondre à une question, mettre dans le code juste après la question :
\begin{verbatim}
\answer[<auteur>]{<réponse>}
\end{verbatim}

N'hésitez pas non plus à classer vos questions par catégorie si ça vous tente \verb!\o/!.

\ask[Clément]{Et alors ça donne quoi ?}

\answer[Clément]{Voilà, on peut dire que, d'une certaine façon et si l'on fait preuve d'une certaine tolérance, ça donne quelque chose de pas trop mal !

Il se trouve que l'on peut même écrire plein de trucs !}

\clearpage

%================DEBUT DOCUMENT=====================
\setcounter{question}{0}


\askg{Comment capter les informations sous forme intelligible par l'Homme ? Autrement dit, comment \og traduire \fg les paquets de façon à pouvoir les interpréter ?}

\answer[Clément]{A priori, une grande partie des informations non cryptées et captées par WireShark sont intelligibles. Si l'on veut savoir comment rendre nous-mêmes intelligible un paquet, peut-être voir de ce côté.}


\askg{Comment sont codés les transferts de données suivants ?

SMS, MMS, mails, web, Bluetooth, envoi de fichiers, etc.}

\askg{Comment envoyer du code à un appareil dans le but de le faire exécuter ? Quel type de paquet peut faire cela, comment faudrait-il le coder ?}

\askf{Quels sont les commandes les plus utilisées en git?}





\end{document}

Ceci est useless